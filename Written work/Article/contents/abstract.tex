\begin{abstract}
The purpose of this project is to investigate the time domain in the automatic scoring of sleep stages by combining a convolutional neural network (CNN) with a recurrent neural network (RNN). 
The raw electroencephalographic signals have been transformed into visual interpretable images by using multi-taper spectral analysis. The six sleep stages are represented in the transformed images with different visuel patterns. By learning visual patterns, it is possible to automatically classify the current stage of the sleep. 
%  \citetitle{main_ar} by \citeauthor{main_ar}
The foundation of this project is based upon the article by A. Vilamala et al. (2017) \cite{main_ar}. Their procedures have been re-created in order to create the baseline. The baseline is then compared to a modified network, which combines a CNN and a RNN.
%Due to the experimental setup and the network implementations in this project, the achieved results are not fully comparable with \cite{main_ar}.

There has been applied bootstrapping in order to find the average performance metrics of the two networks. However, the extended model does not outperform the baseline model.

\end{abstract}
\begin{keywords}
Convolutional Neural Networks, Recurrent Neural Networks, Sleep Stage Scoring, Computer Vision and Pattern Recognition.
\end{keywords}