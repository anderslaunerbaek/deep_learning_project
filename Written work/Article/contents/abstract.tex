\begin{abstract}
The purpose of this paper is to investigate the time domain in automatic scoring of sleep stages by combining a recurrent neural network with a convolutional neural network. 
The raw polisomnography signals have been transformed into visuel interpretable images by using multitaper spectral analysis. The six different sleeping stages are represented in the transformed images with different visuel patterns. By learning those visuel sleeping pattern, it is possible to automatically classify the current stage of the sleep. 
%  \citetitle{main_ar} by \citeauthor{main_ar}
The basis of this project are based upon the article by \cite{main_ar} and their produces has been re-created in order to create the base line. The base line are here compared to an extended model, which combines a convolutional neural network and a recurrent neural network.
Due to the experimentally setup and the implementations of the networks in this project, the results are not comparable with the archived results in \cite{main_ar}.

The performances of the two models are close to similar. The extended model does not out perform the base line model.


\end{abstract}
\begin{keywords}
Convulutional Neural Networks, Recurrent Neural Networks, Sleep Stage Scoring, Computer Vision and Pattern Recognition.
\end{keywords}