\section{Materials and Methods}
\label{sec:materials_and_methods}

 As a requirements for this project, the professor and assistant teahters needs to have acces to the running code which produces the results within this paper. The two bullets below links to what you need in order to reproduce the presented results.
 \begin{itemize}
 	\item Github: \href{https://github.com/anderslaunerbaek/Deep\_Learning\_Project.git}{Deep\_Learning\_Project.git}
 	\item DTU SharePoint: \href{https://dtudk-my.sharepoint.com/personal/s160159\_win\_dtu\_dk/\_layouts/15/guestaccess.aspx?docid=093aa4dcaee0b4e3aa18b0ee67061a678&authkey=AbdnyuYwQUWn0BDEPeDn1Mg&e=b79a063cd72d43a9b7db88f8b8fd1b06}{Data\_dicts\_and\_Code\_models.zip}
 	\item DEMO: \href{https://github.com/anderslaunerbaek/Deep_Learning_Project/tree/master/DEMO}{master/DEMO}
 \end{itemize}

\subsection{Image Creation}
There have been applied multi-taper spectrum analysis in order to turn the EEG signals into images. A given image represents an epoch in seconds of the complete recorded EEG signal along the first axis. The second axis represents the spectrum for the rhythmic components of interest, mentioned above. The third axis (the color) represents the amplitudes of the rhythmic components to a given time. 

The WFDB Toolbox (\cite{matlab}) for Matlab have been used to download, preprocess and transform the EEG signals into images. The applied script\footnote{Git repo: "Code/2. from\_edf\_to\_pic.m"} for this process has been provided by the supervisor. 
The multi-taper spectrum analysis which estimates the images, is not within the scope of the project. The hyperparameters, such as the duration (in $\left[s \right]$) of an epoch, number of multi-tapers, frequency resolution (in $\left[Hz \right]$), etc., within the image estimation process have been decided to remove from possible hyperparameter in this project. This ensures that the results of the baseline model, in this project, can be comparable with the main article \cite{main_ar} and keep the correct focus. 

The Matlab toolbox are able to download the data set of interest and the data which have been used in this projects consists of PSG recordings for 20 Subjects. The Subjects have been monitored for two nights except Subject 20. There is $38211$ images after the preprocessing of the EEG signals. All images have labeled-values which employs a supervised learning approach. Table \ref{tab_class_balance} illustrates how the labels of $38211$ images are distributed for the sleeping stages.

\begin{table}[th!]
\begin{tabular}{l|llllll}
Sleep Stage & W & N1 &  N2& N3 & N4 & R \\\hline
Dist. (in \%) &12 &7&46&9&6&20
\end{tabular}
\caption{This table summerises the aggregates the distribution of the labels for all 20 Subjects. The distribution of the labels illustrates the sleep stages of subjects during the recordings.}
\label{tab_class_balance}
\end{table}

In order to create a state of the art sleep stage classifier, WE need to consider methods to balance the six classes prior to training of the models.



\subsection{Neural Network Architectures}

The architecture for the base line model is the VGGNet 16 \cite{main_ar, VGGnet16} which is a 16-layer network composed of following operations.

\begin{itemize}
\item The convolutional operations uses a kernel of $3x3$, which is the smallest possible configuration to capture the notion of left/right, up/down and center. The kernel has a stride of $1$, which is the number of pixels the kernel slide at the time.
The third hyperparameter within the convolutional operation is the padding. This operation uses padding so the spatial resolution is preserved. This layer performs several linear activations by the kernel, those activations goes through a selected non-linear, ReLU, activation function.
\item The max pooling layer reports the maximum non-linear activation value within its rectangular neighborhood. The rectangular neighborhood for this network is $2x2$. The max pooling operation has a stride of $2$. The spatial max pooling "decreases the resolution of image" and helps make the representation invariant to small translations of the output from the previous convolution operation \cite[sec. 9]{dl_book}.
	\item The last three layers of the VGGNet 16 network is fully connected. The first two have 4096 channels. The last fully connected layer performs a six-channel classification, due to the six classes in this project.
	\item The final layer of the VGGNet 16 network is the Softmax activation. This activation function is used to calculates the probabilities associated for each class \cite[eq. 4.1]{dl_book}.  
\end{itemize}
The VGGNet 16 is an acknowledged deep convolution network suitable and well performing network for image recognition in several case studies \cite{VGGnet16, stand_cnn_notes}.  
The standard input layer takes an $224x224$ RGB image.


\subsubsection{Convolutional Neural Network}

The baseline model for this project is given in equation \ref{eq_vgg16_cnn}\footnote{See implementation in Git repo.: "./Code/master.py"}. 
\begin{equation}
\begin{aligned}
net &= c_{2,64}mc_{2,128}mc_{3,256}mc_{3,512}mc_{3,512}m c_{7,f} d c_{1, f} dc_{1,o}	\\
%&\downarrow\\
logits &= tf.squeeze(net, [1, 2])\\
probs &= tf.nn.softmax(logits)
\end{aligned}
\label{eq_vgg16_cnn}
\end{equation}
where $c_{k,l}$ is a convolutional layer, $k$ is the number of receptive layers and $l$ is the number of channels. $c_{1,f} = c_{1,4096}$ and $c_{1,o} = c_{1,6}$.
$m$ is the max pooling. $d$ is a dropout operation, which keeps  $p = 50\%$ of the total connection between the previous and next layer. 

The TF implementation of the VGGNet 16 is heavily inspired by \cite{git_vgg16}. There have been added few modifications to the compared to the initial VGGNet 16 \cite{VGGnet16} in this implementation. 
The implementation of the VVG16net has transformed the fully connected layers into convolutional 2D layers. It works identically if you change size of the kernel. But then it is necessary to the squeeze the second last layer ($c_{1,o}$) into a 2D-tensor before applying the Softmax activation.
Beside this, there have been included L2-regularization, with a weighted decay of $0.0005$, between the convolutional operations and included dropouts between the fully connected layers. The modifications are included in eq. \ref{eq_vgg16_cnn}.
  
 The L2-regularization and dropout operations prevents overfitting in this deep complex network. 

\subsubsection{Recurrent Neural Network}
It is possible to understand sleep as a sequence of different stages. By learning the transitions between those stages may improve the classification of the sleep stages.
Recurrent neural networks (RNN) is the approach to model sequences and the most effective RNNs is called gated RNNs \cite[sec. 10.10]{dl_book}. 
The long short-term memory (LSTM) network is a model which comprehend the issues of vanishing or exploding gradients over time. The LSTM cell has a internal recurrence (self-loop), which produces paths where the gradient can flow for long durations.  

The main reason for implementing a sequential network is to learn the transitions rules between the sleeping stages.

The first approach to learn the transitions rules, is to modify the baseline model and use its extracted feature as input for the LSTM. A second approach could be to use several LSTM cells with regularization in between. A third approach could be to redefine the problem, discritize the EEG-signals and feed the LSTM with a 1D-tensor. 

Due to limited time only the first approach have been considered. The network which combines the CNN and LSTM cell will further on be mentioned as the RNN.

There are several approaches to feed the LSTM cell. The chosen approach has been inspired by \cite{git_rnn_cnn_1,43455}, where the the LSTM acting on each of the generated feature maps.
Equation \ref{eq_vgg16_rnn}\footnote{See implementation in Git repo.: "./Code/rnn.py"}.  illustrates the implementation of the RNN. The first convolutional layers are similar to the VGGNet 16 model. The LSTM cell is feed by the final convolutional layer. The input shape to the LSTM cell is static, which can cause discarding of the final mini-batch.

\begin{equation}
\begin{aligned}
net &= c_{2,64}mc_{2,128}mc_{3,256}mc_{3,512}mc_{3,512}m lstm_{7,512} d_p fc_0\\
%&\downarrow\\
logits &=  tf.reshape(net, [batch, -1])\\
logits &= slim.layers.fully_connected(logits, 6) \\
probs &= tf.nn.softmax(logits)
\end{aligned}
\label{eq_vgg16_rnn}
\end{equation}
where the notations from eq. \ref{eq_vgg16_cnn} are the same and the $lstm_{7,512}$ is the LSTM cell   which takes the tensor shape (batch x 7 x 7 x 512). $d_p$ is the dropout operation, which keeps  $p = 50\%$ of the total connection between the previous and next layer. $fc_0$ is the final fully connected layer, which represents the highest order feature maps for the six classes.

The implementation of LSTM cell is heavily inspired by \cite{git_lstm}. This implementation takes the input tensor and returns the sate tensor. Both are 4D-tensors (batch x height x width x channels).

As illustrated in eq. \ref{eq_vgg16_rnn}, there have been apllied an dropout operation on the output tensor from the LSTM cell. Then reshaped in order to feed a fully connected layer, which produces easy separable higher-order feature representations for the six classes discriminate. The final fully connected layer is activated with the non-linear Softmax function.

\subsection{Network Visualization}

There exits many of ways visualizing a neural network w.r.t. its input data in order to better understand how the network interprets different inputs.  A common approach in the literature is to visualizing the activations along its propagation through the network. The example from \cite{stand_cnn} gives a great intuition about how the different operations in each layer affect the input data along is propagation though the network. 

Another approach using the main article (\cite{main_ar}) is to calculate sensitivity maps, which determines the relative importance to every input feature $j$. $j$ represents the third dimension in the image which is the values of the RGB color.
The sensitivity maps is created by calculating the gradients w.r.t. the loss of an input image and its label. The mathematically expression are given in equation \ref{eq_1}.
\begin{equation}
s^{\left(j\right)} = \frac { 1 }{ N  } \sum _{ n=1 }^{ N } \left| \frac { \partial L\left( f\left( x \right) ,t \right)  }{ \partial x^{ \left( j \right)  } }  \right| _{x=x_n}
%s^{\left(j\right)} = \frac { 1 }{ N^{\left(j\right)}  } \sum _{ n=1 }^{ N^{\left(j\right)} } \left| \partial L\left( x_n^{\left( j \right)} ,y_n^{\left( j \right)} \right)   \right|
\label{eq_1}
\end{equation}
where $L$ is the loss w.r.t. the input image $f\left(x\right)$ and its label $t$. 

By applying the function $s^{\left(j\right)}$ for an image, it is possible to see the relative importance in determined by the network for each feature $j$. 
Figure \ref{fig_1_21}-\ref{fig_1_36} illustrates the sensitivity maps for each sleep stage for the CNN and RNN respectively. The Sensitivity maps have been created for subject 1.



\subsection{Hyperparameter}




dropout 
learning rate
weights\_regularizer=slim.l2\_regularizer(weight\_decay),


